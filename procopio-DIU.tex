\documentclass[12pt]{article}

\usepackage[font=footnotesize]{caption}
\usepackage{float}
\usepackage{epsf}
\usepackage{epsfig}
\usepackage{subfigure}
\usepackage{latexsym}
\usepackage{color}
\usepackage{wrapfig}
\usepackage{topcapt}
\usepackage{multirow}
\usepackage{tabularx}
\usepackage{hyperref}
\usepackage{xcolor}
\usepackage{mdwlist}
\usepackage{amsmath, amsfonts, amssymb}
\usepackage{algorithmic}
\usepackage{fancyhdr}
\usepackage{url}
\usepackage{multirow}
\usepackage[bottom]{footmisc}
\usepackage{afterpage}
\usepackage{eurosym}
\usepackage{enumitem}
\usepackage[utf8]{inputenc}
\usepackage[english]{babel}
\usepackage{microtype}

\usepackage{amsmath, amsfonts, amssymb}
\usepackage{algorithmic}
\usepackage[hmargin=1in,vmargin=1.3in]{geometry}
\usepackage{url}
\usepackage{multirow}
\usepackage[ruled,noline,linesnumbered]{algorithm2e}
\usepackage{afterpage}


\widowpenalty=1000
\clubpenalty=1000

\input{epsf}

\def\nas{{NASA\ }}
\def\nase{{NASA}}
\def\inst{{NASA Ames Research Center\ }}
\def\inste{{NASA Ames Research Center}}
\def\univ{{UPorto\ }}
\def\unive{{UPorto}}
\def\univn{{NTNU\ }}
\def\univne{{NTNU}}
\def\vig{{UVigo\ }}
\def\vige{{UVigo}}
\def\ldeo{{LDEO\ }}
\def\ldeoe{{LDEO}}
\def\rax{{\texttt{RAX\ }}}
\def\raxe{{\texttt{RAX}}}
\def\rx{{\texttt{T-REX\ }}}
\def\rxe{{\texttt{T-REX}}}
\def\ls{{LSTS\ }}
\def\lse{{LSTS}}
\def\sml{{SmallSat\ }}
\def\smle{{SmallSat}}
\def\mba{{MBARI\ }}
\def\mbae{{MBARI}}
\def\mer{{\texttt{MER\ }}}
\def\mere{{\texttt{MER}}}
\def\rpe{{REP}}
\def\rp{{REP\ }}
\def\pro{{\textmd{PROCOPIO\ }}}
\def\proe{{\textmd{PROCOPIO}}}

\newlength{\doublespacelength}
\setlength{\doublespacelength}{\baselineskip}
\addtolength{\doublespacelength}{0.5\baselineskip}
\newcommand{\doublespace}{\setlength{\baselineskip}{\doublespacelength}}

\newlength{\singlespacelength}
\setlength{\singlespacelength}{\baselineskip}
\newcommand{\singlespace}{\setlength{\baselineskip}{\singlespacelength}}


\newlength{\savedspacing}
\newcommand{\savespacing}{\setlength{\savedspacing}{\baselineskip}}
\newcommand{\restorespacing}{\setlength{\baselineskip}{\savedspacing}}

\setlength{\parskip}{0pt}
\setlength{\parsep}{0pt}
\setlength{\headsep}{0pt}
\setlength{\topskip}{0pt}
\setlength{\topmargin}{0pt}
\setlength{\topsep}{0pt}
\setlength{\partopsep}{0pt}

\newcommand{\kc}[1]{{\color{red}{#1}}}
\newcounter{quotenumber}

\newenvironment{numquote}{%
    \begin{enumerate}%
     \setcounter{enumi}{\value{quotenumber}}%
     \color{darkgray}
    \item \begin{quote}%
}{%
    \end{quote}%
    \setcounter{quotenumber}{\value{enumi}}
    \end{enumerate}%
}%

\makeatletter
\def\myitem{%
   \@ifnextchar[ \@myitem{\@noitemargtrue\@myitem[\@itemlabel]}}
\def\@myitem[#1]{\item[#1]\mbox{}}
\makeatother



\newcommand\blankpage{%
    \null
    \thispagestyle{empty}%
    \addtocounter{page}{-1}%
    \newpage}

\setcounter{secnumdepth}{0} 



\fancyhead[L]{}% empty left
\fancyhead[R]{ % right
\includegraphics[scale=0.5]{fig/sift.png}\includegraphics[width=0.95in,height=0.45in]{fig/uporto.png}\includegraphics[height=0.45in]{fig/vigo.png}\includegraphics[height=0.45in]{fig/ldeo.jpg}
}
\pagestyle{fancy}


\begin{document}

\vspace*{1cm}
\begin{center}
  {\large \bf{\proe}: A Portable Robotic Observatory for Diagnosing Coastal Ocean Health for Monitoring}\\
  % \today
\end{center}

% \vspace*{0.5cm}

\subsection{Summary}


Our oceans host enormous biodiversity, provide multiple ecosystem
services, sustain vibrant economies, and play a significant role in
climate regulation, but are threatened by human activity and climate
change.  We need a \textbf{sustained}, \textbf{persistent}, and
\textbf{affordable} data gathering capability to help us understand
and monitor how key processes such as acidification, hypoxia, toxic
blooms, pollution and erosion (amongst others) are impacting global
ocean sustainability and stewardship.  In coastal regions, this is
especially important, because these areas mediate most of the
interactions between a significant percentage of the world population
and the oceans. 

Urban population growth has exacerbated the pressures on the coastal
ecosystem.  For example, resultant toxic blooms and oxygen depletion
have had deleterious effects on fisheries and other critical resources
that coastal populations depend on, while also impacting human
health. Furthermore, extreme weather events induced by climate change
will only hasten the worsening of water quality in these areas because
of enhanced runoff, coastal erosion and storm surges. An integrative
sea management approach and the protection of natural capital and
marine ecosystem resources can only be achieved with the help of
coordinated observations from space, aerial, surface and underwater
robots guided by Artificial Intelligence (AI) while providing
continual and reliable oceanographic data.

Many large telescopes point toward the heavens, but no such
observational system exists for looking at and into our oceans.  Our
mission is to build a portable, robotic observatory for observing and
managing the health of our endangered coastal waters which can be
rapidly deployed anywhere in the world
(Fig. \ref{fig:mega-cities}). The implications for this concept are
dual use, including for security and DoD applications. 

\subsection{The Idea}

\pro (A Portable Robotic Observatory for Coordinated Oceanographic
Observations) will be a modular system with bespoke approaches related
to water quality in the world's coastal zones with mega-cities. It
will integrate state-of-the-art hardware including a small satellite
(\smle's) constellation, in-situ air, surface and underwater vehicles
with software to control and visualize the information gathered. With
frequent revisit times over a region by a constellation of \smle's,
coupled with latest smart and adaptive AI techniques, robots can
provide opportune solutions in near real-time. 

% robotic technologies reduces
% deployment time to provide opportune solutions and consequently,
% leverages the latest techniques in AI, Robotics and software
% engineering.


% It
% will provide information for water quality measurements suitable for
% lay persons who can obtain and interpret near real-time (hours) data
% visualized at spatial and temporal scales to provide actionable
% information to deal with coastal pollution, erosion, toxic waters and
% sediment laden plumes.

Stakeholders across governments, industry, science, nonprofits and
citizenry will make use of layered views ranging from basic visuals to
the complex queries needed for effective management of resources and
increased scientific knowledge.


Ocean models will be our tools to integrate the information collected from the multiple platforms including underwater vehicles and remote sensing and to predict the evolution of ocean conditions. This integration will provide a unified picture of the whole water volume capturing the diversity of spatio-temporal scales of ocean variability. The model-data assimilated fields will in turn allow us to both improve our understanding of the complex ocean flows and reach the needed enhanced predictive capability.  

\begin{figure}[H]
  \centering
  \includegraphics[scale=0.135]{fig/mega-cities-toxic-1.jpg}
  \caption{Coastal zones are impacted by natural and human-induced
    activities leading to oxygen depleted 'dead' zones, toxic algal
    blooms, pollution, coastal erosion. \pro will use an ensemble of
    small satellites, aerial, surface and underwater vehicles to
    observe the coastal ocean over sustained periods of time to
    provide real-time warning and situational awareness of impactful
    changes to human health, biodiversity and ecosystem services. Note
    the terms UAV: unmanned aerial vehicle, ASV: autonomous surface
    vehicle, AUV: autonomous underwater vehicle, CODAR: Coastal
    High-Frequency (HF) Radar.}
    \label{fig:mega-cities}
\end{figure}

% Natural events such as storm surges, tsunamis and upwelled waters
% impact such mega-city coastal communities in addition. Turbulence in
% the upper water-column with potential injection of nutrients, either
% from the benthic or open ocean waters, can often result in sudden
% outgrowth of harmful algal blooms, or stirring up human-induced
% pollution in such coastal areas.

% In both human and nature induced events, the resulting mix can make it
% unsafe for any form of human activity often with no obvious and
% expected visible sign of near and present danger to local communities.
% However, the consequences can reverberate with mass scale die-off of
% marine life, poisoned shellfish and coastal wildlife and as well as
% causing neurological damage or fatalities to the human population on
% consuming seafood or using beaches for recreation.

% Current forms of monitoring are based on sensors (if present) spaced
% well apart, periodic human-made measurements that can be impacted by
% harsh weather and which typically sub-sample such dynamic coastal
% environments.

% \pro will integrate state-of-the-art hardware including a small
% satellite (\smle) constellation, in-situ air, surface and underwater
% vehicles with software to control and visualize information derived
% from these assets.  The use of \smle's and smart robotic technologies
% reduces deployment time to provide opportune solutions and
% consequently leverages the latest techniques in AI, Robotics and
% software engineering. Coordinated perspectives across different
% synoptic spatial and temporal scales in turn will provide a
% hyper-realistic situational assessment to stakeholders including those
% who drive policy making for human health.

% In the report, “Global Marine Trends 2030”, Lloyds Register predicts
% that by 2030, the coastal ocean will be “almost unrecognizable”. As we
% enter the United Nations “Decade of the Ocean”, \pro will broaden and
% deepen knowledge that will aid and augment global ocean sustainability
% and stewardship, and the management of our Urban Seas.

\subsection{Why now?}

With the onset of a climate crisis, the oceans are changing rapidly in
ways we do not understand. In the report, ``Global Marine Trends
2030'', Lloyds Register predicts that by 2030, the coastal ocean will
be ``almost unrecognizable''. There is an urgent need to develop and
deploy new smart observational methods to provide information at
scales that matter to the 600 million people living along the coast
within 10 meters of the sea level.  Predicting change and providing
early warning of hazardous events is essential for the well-being of
an increasingly vulnerable coastal ecosystem. It is also in line with
the goals of the 2021-2030 UN Decade of Ocean Science for Sustainable
Development.

% The oceans cover more than 70\% of the earth's surface. The base of
% the human food-chain starts with tiny phytoplankton which generate the
% oxygen for every other breath we take.  With the onset of a climate
% crisis, the oceans are changing rapidly in ways we do not
% understand. There is an urgent need to develop and deploy new smart
% observational methods to provide information at scales that matter to
% the 600 million people living along the coast within 10 meters of the
% sea level.  Predicting change and providing early warning of hazardous
% events, including poor water quality, tainted fish stocks and
% intensifying coastal erosion, is essential for the well-being of an
% increasingly vulnerable coastal ecosystem. It is also in line with the
% goals of the 2021-2030 UN Decade of Ocean Science for Sustainable
% Development.

% By leveraging rapid advances in technology, \pro will field an
% innovative system of small satellites and robust autonomous in-situ
% platforms for obtaining unprecedented views of coastal oceans and
% atmospheric and land interfaces. It will aid in the understanding and
% monitoring of coastal waters so that they can be explored and utilized
% in a sustainable and informed manner.

% \pro will leap-frog the traditional incremental and siloed methods in
% ocean observation by leveraging modern computational methods in data
% science, autonomous robotics and smart sensors. The density and
% diversity of observations will change by an order of magnitude, the
% temporal scales of coastal observations will change from weeks (for
% traditional shipboard sampling) or days (for existing satellite data)
% to hours and minutes with the provision of real-time information.


\vspace*{0.1cm}
\subsection{What is the novelty?}

\pro is different from traditional methods for observing the coastal
ocean, which are inefficient, not cost-effective, too sparse in space,
too sporadic in time or too localized. There is poor integration
between the various measurements, especially between those made
in-situ and those made by satellites to produce actionable knowledge.
% for 21\textsuperscript{st} century decision making.

\pro leap-frogs current methods by delivering predictive modeling,
learning and analytical capabilities, which are supported by AI and
visualization techniques that are non-existent in other interventions.
With \pro, the density and diversity of observations will change by an
order of magnitude, the temporal scales of coastal observations will
change from weeks (for traditional shipboard sampling) or days (for
existing satellite data) to \emph{hours and minutes} with the
provision of real-time information. Techniques in AI will adapt the
information depending on the kind of user, from well-informed
scientists, to the lay person curious about how beach conditions might
impact her leisure. 

% Using AI will adapt the kind of information depending on the kind of
% user. 

In the process of providing actionable knowledge, \pro will enable new
modes of management and new understanding about coastal ocean
processes in ways simply not possible before. \pro will allow citizens
to develop critical understanding of the rapid change taking place in
their Urban Seas and to ‘connect the dots’ between human activity and
the effect on the environment around them. Citizen scientists will be
engaged in generating new observations and be able to derive new
knowledge about how ocean processes work. Scientists will be able to
pose (and answer) new questions that could not have been asked
before. And policy makers will have the tools to make informed
decisions in time scales that matter, while developing truly
integrative policies on ocean sustainability and
stewardship. % \pro will serve as a replicable blueprint for Ocean
% observation in targeting integration, synthesis, cost-effectiveness
% and scalability.

% While some coastal ocean observatories use a limited number of robotic
% assets or remote sensing data, \pro is unique in the range and
% diversity of how these sensors are deployed, how data is integrated
% and synthesized, and how citizen engagement is used to improve the
% value of the output.  Additionally \pro provides an integrative
% open-source framework to connect robots, services and users in a
% seamless manner that is both scalable and replicable, providing a
% blueprint for other initiatives worldwide. It leap-frogs current
% methods by delivering 21st century predictive modeling, learning and
% analytical capabilities, which are supported by AI and visualization
% techniques that are non-existent in other interventions.


\subsection{Milestones and Deliverables}

\begin{itemize}[noitemsep,topsep=0pt,parsep=0pt,partopsep=0pt]

\item Architectural design of the system with a focus on software
  integration, building hardware and design and testing of Machine
  Learning systems for ocean model prediction (Year \textbf{1})

\item Use of existing remote sensing data products (e.g. ESA and NASA
  data products), integration of ocean models and building of AI-based
  adaptive control systems for aerial, surface and underwater
  vehicles.  (Years \textbf{1--2})

\item Incremental at-sea testing of adaptations of robotic vehicles
  and integration of control with ocean model predictions (Years
  \textbf{2--3}) 

\item Demonstration of the integrative software system using existing
  aerial, surface and underwater vehicles from the Univ. of Porto and
  targeting a single extensive use-case (e.g. from aquaculture, oil \&
  gas, others) to monetize this effort (Year \textbf{3})

\item Upscope demonstration to include larger data sources for
  physical ocean properties, including from buoys and synthetic data
  sources (e.g. surf forecasts). If \sml budget permits, begin design
  and build of satellite elements including payload elements (Years
  \textbf{3--4})

\item \sml launch and operation begins. Validation of satellite
  payloads and calibration of sensor performance (Years \textbf{4--5})

\item Pursue European Union and other large funding schemes to fund a
  larger constellation of \smle's to demonstrate the full capability
  in a coastal meso-scale ($\sim 50$ Km\textsuperscript{2}) ecosystem
  (Years \textbf{3--5})

\end{itemize}

\subsection{Resources Needed}

The \pro team comes ready with the aerial, surface and underwater
vehicle platforms, together with the extensive suite of software to
provide coordinated observations in the coastal ocean. We will build
custom sensors keyed towards important ocean variables integrated into
a 'train' of \sml platforms.  Such a system working synchronously with
in-situ robots will provide a clear consistent set of data
products. This data will be integrated to provide actionable
information to policymakers on the ground, as also society in general.

We estimate the total project cost to be $\sim \$63.5$ Million over a
period of 5 years. As milestones are met in the first two years, and
the integrated software can be demonstrated on targeted use-cases,
\pro is likely to attract funding from public and private
sources. Consequently, the project can also be funded in incremental
steps:

\begin{itemize}[noitemsep,topsep=0pt,parsep=0pt,partopsep=0pt]

\item an initial focus on the software build, integration and test
  with available robotic vehicles in small scale demonstrations $\sim
  \$5$--$\$10$ Million for 2 years.

\item acquisition of robotic vehicles, buoys, floats and a range of
  sensors as payloads for these in-situ vehicles, their integration,
  deployment and demonstration at increasingly larger spatial and
  temporal scales for $\sim \$20$--$\$30$ Million for 2 years.

\item acquisition of funds for a suitable at-scale design, build,
  test, launch and operation of a \sml constellation with a range of
  scientific payloads for biological and physical oceanographic
  measurements for $\sim \$25$ Million for 2 years.

\end{itemize}  

Incremental build and evaluation of this concept can allow us to
attract a wide range of public and private sponsors in the US and
Europe.  Equally, we will consistently work with our collaborators in
the Portuguese government to leverage expensive ship time for testing,
and other potential in-kind contributions from Portuguese and Spanish
resources.

For long-term operation and viability of this system, multiple
outcomes can be envisioned. First, with the experience garnered in
testing and fielding the system, a commercial spin-off of all or parts
of the technology could be very possible. If parts of the technology
could be monetized and spun off to other companies, \pro can then hold
the IP while continuing to work on research outcomes after the 5 year
term. Second, the project can itself look for contracts from
mega-cities and governments or their agencies to provide a
software-as-a-service model and be able to subsist as a not-for-profit
enterprise with unique expertise. Should other private or public
funding sources be available, those would also be carefully evaluated
at this time.


\subsection{Governance}

The governing board of \pro will consist of prominent strategic
advisors from the US and Europe including stakeholders and funders. In
addition, the project principals will be aided and advised by a
scientific advisory board consisting of technologists, ocean going
scientists, ecologists and policy makers from the US, Europe and
targeted coastal states.

\subsection{The Team}

\proe’s inter-disciplinary team of seasoned researchers (see bio's
below) from the universities of Columbia/US, Porto/Portugal and
Vigo/Spain have worked in all the major oceans, fielded tens of robots
at sea simultaneously, designed/built/flown and operated multiple
\smle's and complex systems in the deep sea and deep space.


% \newpage
\vspace*{0.5cm}
\section{Brief Bios of the Principals}



\parbox{6.25in}{
\begin{wrapfigure}{r}{0.45\textwidth}
  \centering
  \includegraphics[width=.75\linewidth]{fig/FAguado.jpg}
\end{wrapfigure}
\textbf{Fernando Aguado} is an Associate Professor at the University
of Vigo. He was the principal investigator (PI) of the Xatcobeo
cubesat, the first Galician satellite and first Spanish Cubesat. He
was also the PI of HUMSAT-D and sector B of Serpens (both developed
within the Basic Space Technology Initiative of UN with the support of
ESA) and the LUME-1 \smle. He has coordinated the design, manufacturing
and integration of various Cubesats for maritime and airplane tracking
applications as well as for machine to machine communications.
\\
\textbf{email: }\emph{faguado@tsc.uvigo.es} \\
\textbf{Web: }\url{https://bit.ly/3avJlwX}\\
}

\vspace*{+0.1in}

\parbox{6.25in}{
\begin{wrapfigure}{r}{0.45\textwidth}
  \centering
  \includegraphics[width=.75\linewidth]{fig/Krajan.png}
\end{wrapfigure}
\textbf{Kanna Rajan}\footnotemark\footnotetext{Principal
point of contact} holds a visiting faculty appointment at the \univ
in autonomous systems. He spent 10 years at \inst where his software
was responsible for the command/control of the 1999 New Millennium
Deep Space 1, 65 Million miles from Earth and the 2003 Mars
Exploration Rovers mission on the Red Planet. In 2005 he moved to \mba
and built the only AI group in marine robotics. His field work and
publications have been in highly ranked peer-reviewed publications
including \emph{Science}, Intnl. Journal of Robotics Research, the AI
Journal and Journal of Field Robotics. He has participated in
scientific oceanographic cruises in the Pacific and Atlantic.
\\
\textbf{email: }\emph{Kanna.Rajan@sift.net}\\
\textbf{Web:}\url{https://bit.ly/3bfCnwP}
 }

\newpage
\vspace*{0.5cm}

\parbox{6.25in}{
\begin{wrapfigure}{r}{0.45\textwidth}
  \centering
  \includegraphics[width=.75\linewidth]{fig/JBS.jpg}
\end{wrapfigure}
\textbf{Jo\~ao Sousa} is a Professor at the Faculty of Engineering,
Univ. of Porto, Portugal and the head of the Underwater Systems and
Technology Laboratory (\lse). The lab has pioneered the design,
construction and deployment of networked underwater, surface and air
vehicles for applications in ocean sciences and defense and is at the
vanguard of operations of coordinated aerial, surface and underwater
vehicles. The lab designed the award-winning Light Autonomous
Underwater Vehicle and the \ls open source software for networked
vehicle systems, and has been key in organizing large scale
experiments, including the annual Rapid Environmental Picture (\rpe)
organized jointly with the Portuguese Navy since 2010. He has
participated in numerous engineering and scientific oceanographic
cruises.
\\
\textbf{email: }\emph{jtasso@fe.up.pt}\\
\textbf{Web: }\url{https://bit.ly/2J6cKCc}}
% {https://lsts.pt/member/jo%C3%A3o-sousa}


\vspace{+0.25in}
\parbox{6.25in}{
\begin{wrapfigure}{r}{0.45\textwidth}
 \centering
  \includegraphics[width=.75\linewidth]{fig/ASub.jpg}
\end{wrapfigure}

\textbf{Ajit Subramaniam} is a Lamont Research Professor at the
Lamont-Doherty Earth Observatory (\ldeoe) of Columbia University, New
York.  He is an oceanographer who uses knowledge of remote sensing,
ocean optics, phytoplankton physiology, biological and physical
oceanography and geographical information systems to better understand
how the marine ecosystem functions and can be managed.  He has worked
for National Oceanic and Atmospheric Administration (NOAA), and held
appointments at the Univ. of Maryland and the Univ. of Southern
California prior to moving \ldeo in 2004. He has served as the Program
Director for the Marine Microbiology Initiative at the Gordon and
Betty Moore Foundation and a program manager in the Biological
Oceanography Program at the U.S. National Science Foundation (NSF). He
has participated in oceanographic cruises in all parts of the world.
\\
\textbf{email: }\emph{ajit@ldeo.columbia.edu} \\
\textbf{Web: }\url{https://www.ldeo.columbia.edu/~ajit/}
}

\end{document}

