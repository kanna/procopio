\documentclass[12pt]{article}

\usepackage[font=footnotesize]{caption}
\usepackage{float}
\usepackage{epsf}
\usepackage{epsfig}
\usepackage{subfigure}
\usepackage{latexsym}
\usepackage{color}
\usepackage{wrapfig}
\usepackage{topcapt}
\usepackage{multirow}
\usepackage{tabularx}
\usepackage{hyperref}
\usepackage{xcolor}
\usepackage{mdwlist}
\usepackage{amsmath, amsfonts, amssymb}
\usepackage{algorithmic}
\usepackage{fancyhdr}
\usepackage{url}
\usepackage{multirow}
\usepackage[bottom]{footmisc}
\usepackage{afterpage}
\usepackage{eurosym}
\usepackage{enumitem}
\usepackage[utf8]{inputenc}
\usepackage[english]{babel}
\usepackage{microtype}

\usepackage{amsmath, amsfonts, amssymb}
\usepackage{algorithmic}
\usepackage[hmargin=1in,vmargin=1.3in]{geometry}
\usepackage{url}
\usepackage{multirow}
\usepackage[ruled,noline,linesnumbered]{algorithm2e}
\usepackage{afterpage}


\widowpenalty=1000
\clubpenalty=1000

\input{epsf}

\def\nas{{NASA\ }}
\def\nase{{NASA}}
\def\inst{{NASA Ames Research Center\ }}
\def\inste{{NASA Ames Research Center}}
\def\univ{{UPorto\ }}
\def\unive{{UPorto}}
\def\univn{{NTNU\ }}
\def\univne{{NTNU}}
\def\vig{{UVigo\ }}
\def\vige{{UVigo}}
\def\ldeo{{LDEO\ }}
\def\ldeoe{{LDEO}}
\def\rax{{\texttt{RAX\ }}}
\def\raxe{{\texttt{RAX}}}
\def\rx{{\texttt{T-REX\ }}}
\def\rxe{{\texttt{T-REX}}}
\def\ls{{LSTS\ }}
\def\lse{{LSTS}}
\def\sml{{SmallSat\ }}
\def\smle{{SmallSat}}
\def\mba{{MBARI\ }}
\def\mbae{{MBARI}}
\def\mer{{\texttt{MER\ }}}
\def\mere{{\texttt{MER}}}
\def\rpe{{REP}}
\def\rp{{REP\ }}
\def\pro{{\textmd{PROCOPIO\ }}}
\def\proe{{\textmd{PROCOPIO}}}

\newlength{\doublespacelength}
\setlength{\doublespacelength}{\baselineskip}
\addtolength{\doublespacelength}{0.5\baselineskip}
\newcommand{\doublespace}{\setlength{\baselineskip}{\doublespacelength}}

\newlength{\singlespacelength}
\setlength{\singlespacelength}{\baselineskip}
\newcommand{\singlespace}{\setlength{\baselineskip}{\singlespacelength}}


\newlength{\savedspacing}
\newcommand{\savespacing}{\setlength{\savedspacing}{\baselineskip}}
\newcommand{\restorespacing}{\setlength{\baselineskip}{\savedspacing}}

\setlength{\parskip}{0pt}
\setlength{\parsep}{0pt}
\setlength{\headsep}{0pt}
\setlength{\topskip}{0pt}
\setlength{\topmargin}{0pt}
\setlength{\topsep}{0pt}
\setlength{\partopsep}{0pt}

\newcommand{\kc}[1]{{\color{red}{#1}}}
\newcounter{quotenumber}

\newenvironment{numquote}{%
    \begin{enumerate}%
     \setcounter{enumi}{\value{quotenumber}}%
     \color{darkgray}
    \item \begin{quote}%
}{%
    \end{quote}%
    \setcounter{quotenumber}{\value{enumi}}
    \end{enumerate}%
}%

\makeatletter
\def\myitem{%
   \@ifnextchar[ \@myitem{\@noitemargtrue\@myitem[\@itemlabel]}}
\def\@myitem[#1]{\item[#1]\mbox{}}
\makeatother



\newcommand\blankpage{%
    \null
    \thispagestyle{empty}%
    \addtocounter{page}{-1}%
    \newpage}

\setcounter{secnumdepth}{0} 


% \usepackage{picins}

\fancyhead[L]{}% empty left
\fancyhead[R]{ % right
  \includegraphics[width=1.5in,height=0.45in]{fig/sift.jpg}
\hspace{+0.4cm}\includegraphics[width=0.95in,height=0.45in]{fig/uporto.png}
\hspace{+0.4cm}\includegraphics[height=0.45in]{fig/vigo.png}
\hspace{+0.4cm}\includegraphics[height=0.45in]{fig/ldeo.jpg}
\hspace{+0.4cm}\includegraphics[height=0.5in]{fig/socib.jpg}
\hspace{+0.4cm}\includegraphics[width=0.90in,height=0.40in]{fig/mit.png}
} \pagestyle{fancy}


\begin{document}

\vspace*{1cm}
\begin{center}
  {\large \bf{\proe}: A Portable Robotic Observatory for Diagnosing
    Coastal Ocean Health for Human Well
    Being}\footnotemark\footnotetext{Principal point of
    contact: \emph{Kanna.Rajan@sift.net}}\\
  % \today
\end{center}

% \vspace*{0.5cm}


% Modelling is related to 2 major actions: data integration (in situ and
% remote) through data assimilation and enhancing forecasting skills.

\vspace*{-0.2cm}
\subsection{Summary}
`

Our oceans host enormous biodiversity, provide multiple ecosystem
services, sustain vibrant economies, and play a significant role in
climate regulation, but are threatened by human activity and climate
change.  We need a \textbf{sustained}, \textbf{persistent}, and
\textbf{affordable} data gathering and assimilating capability to help
us understand and monitor how key processes such as acidification,
hypoxia, toxic blooms, pollution and erosion (amongst others) are
impacting global ocean sustainability and stewardship.  In coastal
regions, this is especially important, because these areas mediate
most of the interactions between a significant percentage of the world
population and the oceans.

Urban population growth has exacerbated the pressures on the coastal
ecosystem.  For example, resultant toxic blooms and oxygen depletion
have had deleterious effects on fisheries and other critical resources
that coastal populations depend on, while also impacting human
health. Furthermore, extreme weather events induced by climate change
will only hasten the worsening of water quality in these areas because
of enhanced runoff, coastal erosion and storm surges. An integrative
sea management approach and the protection of natural capital and
marine ecosystem resources can only be achieved with the help of
coordinated observations from space, aerial, surface and underwater
robots guided by Artificial Intelligence (AI).
% while providing continual and reliable oceanographic data.

Many large telescopes point toward the heavens, but no such
observational system exists for looking at and into our oceans.  Our
mission is to build a portable, robotic observatory for observing,
analyzing and managing the health of our endangered coastal waters
which can be rapidly deployed anywhere in the world
(Fig. \ref{fig:mega-cities}).

\vspace*{-0.2cm}
\subsection{The Idea}

\pro (A Portable Robotic Observatory for Coordinated Oceanographic
Observations) will be a modular system with bespoke approaches related
to water quality in the world's coastal zones with mega-cities. It
will integrate state-of-the-art hardware including a small satellite
(\smle's) constellation, in-situ air, surface and underwater vehicles
with software to control and visualize the information gathered. With
frequent revisit times over a region by a constellation of \smle's,
coupled with latest smart and adaptive AI techniques, robots can
provide systematic and opportune observations in near real-time.

% robotic technologies reduces
% deployment time to provide opportune solutions and consequently,
% leverages the latest techniques in AI, Robotics and software
% engineering.


% It
% will provide information for water quality measurements suitable for
% lay persons who can obtain and interpret near real-time (hours) data
% visualized at spatial and temporal scales to provide actionable
% information to deal with coastal pollution, erosion, toxic waters and
% sediment laden plumes.


\begin{figure}[H]
  \centering
  \includegraphics[scale=0.115]{fig/mega-cities-toxic-1.jpg}
  \caption{Coastal zones are impacted by natural and human-induced
    activities leading to oxygen depleted 'dead' zones, toxic algal
    blooms, pollution and coastal erosion. \pro will use an ensemble
    of small satellites, aerial, surface and underwater vehicles along
    with predictive shore-side ocean models, to observe the coastal
    ocean over sustained periods of time to provide real-time warning
    and situational awareness of impactful changes to human health,
    biodiversity and ecosystem services. Note the terms UAV: unmanned
    aerial vehicle, ASV: autonomous surface vehicle, AUV: autonomous
    underwater vehicle, CODAR: Coastal High-Frequency (HF) Radar.}
    \label{fig:mega-cities}
\end{figure}

Using \proe, stakeholders across governments, industry, science,
nonprofits and citizenry will make use of layered views ranging from
basic visuals to the complex queries needed for effective management
of resources and increased scientific knowledge.  Ocean models in the
cloud, will integrate the information collected from multiple
platforms, robotic vehicles as well as satellite remote sensing, so as
to predict the evolution of oceanographic conditions. This integration
will provide a unified picture of the surveyed water volume capturing
the diversity of phenomenon and connect that to natural ocean
variability. The model-data assimilated from robotic platforms will in
turn allow us to improve our understanding of the complex ocean flows
and increase predictive skill.
% Natural events such as storm surges, tsunamis and upwelled waters
% impact such mega-city coastal communities in addition. Turbulence in
% the upper water-column with potential injection of nutrients, either
% from the benthic or open ocean waters, can often result in sudden
% outgrowth of harmful algal blooms, or stirring up human-induced
% pollution in such coastal areas.

% In both human and nature induced events, the resulting mix can make it
% unsafe for any form of human activity often with no obvious and
% expected visible sign of near and present danger to local communities.
% However, the consequences can reverberate with mass scale die-off of
% marine life, poisoned shellfish and coastal wildlife and as well as
% causing neurological damage or fatalities to the human population on
% consuming seafood or using beaches for recreation.

% Current forms of monitoring are based on sensors (if present) spaced
% well apart, periodic human-made measurements that can be impacted by
% harsh weather and which typically sub-sample such dynamic coastal
% environments.

% \pro will integrate state-of-the-art hardware including a small
% satellite (\smle) constellation, in-situ air, surface and underwater
% vehicles with software to control and visualize information derived
% from these assets.  The use of \smle's and smart robotic technologies
% reduces deployment time to provide opportune solutions and
% consequently leverages the latest techniques in AI, Robotics and
% software engineering. Coordinated perspectives across different
% synoptic spatial and temporal scales in turn will provide a
% hyper-realistic situational assessment to stakeholders including those
% who drive policy making for human health.

% In the report, “Global Marine Trends 2030”, Lloyds Register predicts
% that by 2030, the coastal ocean will be “almost unrecognizable”. As we
% enter the United Nations “Decade of the Ocean”, \pro will broaden and
% deepen knowledge that will aid and augment global ocean sustainability
% and stewardship, and the management of our Urban Seas.
\vspace*{-0.2cm}
\subsection{Why now?}

With the onset of a climate crisis, the oceans are changing rapidly in
ways we do not understand. In the report, ``Global Marine Trends
2030'', Lloyds Register predicts that by 2030, the coastal ocean will
be ``almost unrecognizable''. There is an urgent need to develop and
deploy new smart observational methods to provide information at
scales that matter to the 600 million people living along the coast
within 10 meters of sea level.  Predicting change and providing early
warning of hazardous events is essential for the well-being of an
increasingly vulnerable coastal ecosystem. It is also in line with the
goals of the 2021-2030 UN Decade of Ocean Science for Sustainable
Development.

% The oceans cover more than 70\% of the earth's surface. The base of
% the human food-chain starts with tiny phytoplankton which generate the
% oxygen for every other breath we take.  With the onset of a climate
% crisis, the oceans are changing rapidly in ways we do not
% understand. There is an urgent need to develop and deploy new smart
% observational methods to provide information at scales that matter to
% the 600 million people living along the coast within 10 meters of the
% sea level.  Predicting change and providing early warning of hazardous
% events, including poor water quality, tainted fish stocks and
% intensifying coastal erosion, is essential for the well-being of an
% increasingly vulnerable coastal ecosystem. It is also in line with the
% goals of the 2021-2030 UN Decade of Ocean Science for Sustainable
% Development.

% By leveraging rapid advances in technology, \pro will field an
% innovative system of small satellites and robust autonomous in-situ
% platforms for obtaining unprecedented views of coastal oceans and
% atmospheric and land interfaces. It will aid in the understanding and
% monitoring of coastal waters so that they can be explored and utilized
% in a sustainable and informed manner.

% \pro will leap-frog the traditional incremental and siloed methods in
% ocean observation by leveraging modern computational methods in data
% science, autonomous robotics and smart sensors. The density and
% diversity of observations will change by an order of magnitude, the
% temporal scales of coastal observations will change from weeks (for
% traditional shipboard sampling) or days (for existing satellite data)
% to hours and minutes with the provision of real-time information.


\vspace*{-0.2cm}
\subsection{What is the novelty?}

\pro is different from traditional methods for observing the coastal
ocean, which are inefficient, not cost-effective, too sparse in space,
too sporadic in time or too localized. There is poor integration and
assimilation of multiple data sources especially between those made
in-situ and those made by satellites to produce actionable knowledge.

\pro leap-frogs current methods by delivering predictive modeling,
Machine Learning and AI driven analytical capabilities, augmented by
visualization techniques that are non-existent in other interventions.
The density and diversity of observations will change by an order of
magnitude, the temporal scales of coastal observations will change
from weeks or days to \emph{hours and minutes} with the provision of
near real-time information. Techniques in AI will adapt the
information depending on the kind of user, from well-informed
scientists, to the lay person curious about how beach conditions might
impact her leisure.

% \pro leap-frogs current methods by delivering predictive modeling,
% Machine Learning and AI driven analytical capabilities, augmented by
% visualization techniques that are non-existent in other interventions.
% With \proe, the density and diversity of observations will change by
% an order of magnitude, the temporal scales of coastal observations
% will change from weeks (for traditional shipboard sampling) or days
% (for existing satellite data) to \emph{hours and minutes} with the
% provision of real-time information. Techniques in AI will adapt the
% information depending on the kind of user, from well-informed
% scientists, to the lay person curious about how beach conditions might
% impact her leisure.

In the process of providing actionable knowledge, \pro will enable new
modes of management and new understanding about coastal ocean
processes in ways simply not possible before. \pro will allow citizens
to develop critical understanding of the rapid change taking place in
their Urban Seas and to ‘connect the dots’ between human activity and
the effect on the environment around them. Citizen scientists will be
engaged in generating new observations and be able to derive new
knowledge about how ocean processes work. Scientists will be able to
pose (and answer) new questions that could not have been asked
before. And policy makers will have the tools to make informed
decisions in time scales that matter, while developing truly
integrative policies on ocean sustainability and
stewardship. % \pro will serve as a replicable blueprint for Ocean
% observation in targeting integration, synthesis, cost-effectiveness
% and scalability.

% While some coastal ocean observatories use a limited number of robotic
% assets or remote sensing data, \pro is unique in the range and
% diversity of how these sensors are deployed, how data is integrated
% and synthesized, and how citizen engagement is used to improve the
% value of the output.  Additionally \pro provides an integrative
% open-source framework to connect robots, services and users in a
% seamless manner that is both scalable and replicable, providing a
% blueprint for other initiatives worldwide. It leap-frogs current
% methods by delivering 21st century predictive modeling, learning and
% analytical capabilities, which are supported by AI and visualization
% techniques that are non-existent in other interventions.

\subsection{Milestones and Deliverables}

\pro milestones and deliverables will be along the following lines:

% Overall expenses related to personnel and equipment for a 4 year term
% of project development, testing and initial deployment in
% \ref{fig:expense}. The overall proportion of the costs in
% \ref{fig:exp-pie}

% \begin{table}[!h]
%   \centering
%   \vspace{-0.5cm}
%   \begin{tabular}{|p{1.8cm}|p{13cm}|}\hline 
%     \rowcolor{Gray}
%     \bfseries Year &\bfseries Description \\
%     \hline
%     \textbf{Sept/Oct 2021} & Proof-of-concept field demonstration off of
%                              Portugal, with assimilation of measurements from aerial, surface and
%                              underwater vehicles into oceanographic models with remote sensing
%                              provided by a \nas \sml overflight and other sources.\\  
%     \hline
%     \textbf{1} & Architectural system design with a focus on software
%                  integration, building hardware and design and testing of Machine
%                  Learning for ocean model prediction.\\ 
%     \hline
%     \textbf{1--2} & Use and integration of existing remote sensing data products
%                     (from \esa and \nase), integration of ocean models and building of
%                     AI-based adaptive control systems for aerial, surface and underwater
%                     vehicles. \\  
%     \hline
%     \textbf{1--2} & Incremental demonstration of closing the
%                     prediction-sensing-assimilation loop for dynamic ocean events in the
%                     coastal region.\\
%     \hline
%     \textbf{1--3} & Incremental at-sea testing of adaptations of robotic vehicles
%                     and integration of control with ocean model predictions.\\

%     \hline
%     \textbf{1--3} & Demonstrations of the integrative software system using existing
%                     aerial, surface and underwater vehicles and targeting a single
%                     use-case (e.g. from aquaculture, oil \& gas, others) for
%                     monetization.\\
%     \hline
%     \textbf{3--4} & Upscope demonstration to include larger data sources for
%                     physical ocean properties, including
%                     from buoys and surf forecasts.\\ 

%     \hline
%     \textbf{3--4} & Pursue European Union and other sources to fund a constellation
%                     of \smle's to demonstrate the full capability in a coastal ecosystem.\\ 
%     \hline
%   \end{tabular}
%   \caption{Proposed timeline of milestones and deliverables.}
%   \label{tab:timeline}
%   \vspace{-0.5cm}
% \end{table}

\begin{itemize}[noitemsep,topsep=0pt,parsep=5pt,partopsep=10pt]

\item A Sept/Oct 2021 proof-of-concept field demonstration off of
  Portugal, with assimilation of measurements from aerial, surface and
  underwater vehicles into oceanographic models with remote sensing
  provided by a cubesat that was developed with funding from the Moore
  Foundation as well as other more traditional sources.

\item Architectural system design with a focus on software
  integration, building hardware and design and testing of Machine
  Learning for ocean model prediction (Year \textbf{1})

\item Use and integration of existing remote sensing data products
  (from \esa and \nase), integration of ocean models and building of
  AI-based adaptive control systems for aerial, surface and underwater
  vehicles.  (Years \textbf{1--2})

\item Phased demonstration of closing the
  prediction-sensing-assimilation loop for dynamic ocean events in the
  coastal region (Years \textbf{1--2})

\item Phased at-sea testing of adaptations of robotic vehicles and
  integration of control with ocean model predictions (Years
  \textbf{1--3})

\item Demonstrations of the integrative software system using existing
  aerial, surface and underwater vehicles and targeting a single
  use-case (e.g. from aquaculture, oil \& gas, coastal pollution
  around urban centers, others) for monetization (Year \textbf{1--3})

\item Upscope demonstration to include larger data sources for
  physical ocean properties, including from buoys and surf forecasts
  (Years \textbf{3--4})

% \item \sml launch and operation begins. Validation of satellite
%   payloads and calibration of sensor performance (Years \textbf{4--5})

\item Pursue European Union and other sources to fund a constellation
  of \smle s to demonstrate the full capability in a coastal ecosystem
  (Years \textbf{3--4})
  
\end{itemize}

\begin{figure}[!t]
  \centering
  \includegraphics[scale=0.22]{fig/timeline.jpg}
  \caption{Overview of timeline in steps for \proe. Initial
    demonstration of key concepts with a simple demonstration in a
    field experiment in 2021 would ideally be followed by a systematic
    engineering effort to build the needed software
    infrastructure. Should \sml constellation build/test/launch/flight
    and purchase of hardware be possible, that effort can be
    accommodated with the software build shown.}
  \label{fig:timeline}
\end{figure}

% \noindent
Our first step will be a rapid and simple demonstration of the key
concepts in the September/October 2021 timeframe using existing remote
sensing products including that from an ocean color cubesat funded by
the Moore Foundation for environmental assimilation, and available
robotic platforms, for an experiment that links in-situ measurements
with ocean model assimilation, prediction and intelligent adaptive
sampling. We will use this exercise to demonstrate to key stakeholders
the potential of systematic integration of platforms and sensors, with
models, so as to visualize in detail, generated data products. A more
systematic integration, the focus of this proposal, will require the 4
year effort proposed above. Should sponsorship of the \sml
constellation occur during this phase, the build/test/launch/flight
and integration of these assets with the software will
commence. Subsequent spin-offs and commercialization will need skilled
staff to do the outreach to a range of commercial, governmental and
NGO entities.


% \begin{wrapfigure}{r}{0.45\textwidth}
%   % \vspace{-1cm}
%   \centering
%   \includegraphics[scale=0.35]{fig/sat-progression.pdf}
%   \caption{Progression of deployment for 12 \smle's over a period of 5
%     years.}
%   \label{fig:sat-prog}
%   % \vspace{-0.5cm}
% \end{wrapfigure}


% shows costs associated with the project broken
% down to provide a clear picture of the distributions of proposed
% expenses. The end product will be a software system which will predict
% oceanographic conditions, use the prediction to adaptively place
% in-situ robotic vehicles to sample at the 'right place and right
% time', assimilate the sensed environment in ocean models and provide a
% reiterated prediction.

% Should the \sml part of the project be simultaneously
% executed with the development of software and hardware for in-situ
% assets, the two figures would then be merged. The incremental
% deployment of the \smle's is shown in Fig. \ref{fig:sat-prog}.

% \parbox[t]{\dimexpr\textwidth-\leftmargin}{%
%       \vspace{-2.5mm}
%       \begin{wrapfigure}[10]{r}{0.5\textwidth}
%         \centering
%         \vspace{-\baselineskip}
%         \subfigure[]{\label{fig:insitu}\includegraphics[width=.75\linewidth]{fig/insitu.pdf}}
%         \subfigure[]{\label{fig:sats}\includegraphics[width=.75\linewidth]{fig/sats.pdf}}
%         \caption{Costs and distribution for \ref{fig:insitu} assets and
%           software and \ref{fig:sats} \smle's over a 5 year project term.}
%       \end{wrapfigure}
%     }

% \noindent
Fig. \ref{fig:costs} shows the anticipated cost breakdown to provide a
clear picture of the distribution of the budget. The end product will
be a software system that will predict oceanographic conditions, use
the prediction to adaptively place in-situ robotic vehicles to take
water samples at the ‘right place and right time’, assimilate the
sensed environment in ocean models and provide a reiterated
prediction. 

\begin{figure}[!h]
  \centering
  \subfigure[]{\label{fig:expense}\includegraphics[scale=0.4]{fig/expenses.pdf}}
  \subfigure[]{\label{fig:exp-pie}\includegraphics[scale=0.36]{fig/expenses-pie.pdf}}
  \caption{Budget needs and distribution for a 4-year project term:
    \subref{fig:expense} Total expenses related to personnel and
    equipment for a 4-year term; \subref{fig:exp-pie} Proportions of
    total budget for different uses (4-year averages).}
  \label{fig:costs}
  \vspace{-0.5cm}
\end{figure}

\subsection{Resources Needed}

The \pro team (see biographies below) comes ready with aerial, surface
and underwater vehicle platforms, together with the extensive suite of
communication software to provide coordinated observations in the
coastal ocean at the UPorto and SOCIB. With sensors measuring key
ocean variables, the focus of our effort will be in integrating
multiple data streams, increasing modeling skill with assimilated data
and adaptively targeting the in-situ vehicles to narrow the knowledge
gap of current oceanographic conditions. For this effort we will
continue to rely on existing remote sensing data products for
environmental assimilation. This in turn will provide a clear
consistent set of data products to provide actionable information to
policymakers on the ground, and society at large.

% comes ready with aerial, surface and underwater vehicle platforms,
% together with the extensive suite of communication software to provide
% coordinated observations in the coastal ocean at the \univ and
% \soce. With sensors measuring key ocean variables, the focus of our
% effort will be in integrating multiple data streams, increasing
% modelling skill with assimilated data and adaptively targeting the
% in-situ vehicles to narrow the knowledge gap of current oceanographic
% conditions. For this effort we will continue to rely on existing
% remote sensing data products for environmental assimilation. This in
% turn
% % We will integrate custom sensors keyed towards important ocean
% % variables integrated into a 'train' of \sml platforms.  Such a system
% % working synchronously with in-situ robots
% will provide a clear consistent set of data products to provide
% actionable information to policymakers on the ground, as also society
% in general.

We estimate the total project cost to be $\sim\$8.8$ Million over a
period of 4 years. As milestones are met in the first two years, and
the integrated software can be demonstrated on targeted use-cases,
\pro is likely to attract funding from public and private
sources. % Consequently, the project can also be funded in incremental
% steps:

% \begin{itemize}[noitemsep,topsep=0pt,parsep=0pt,partopsep=0pt]

% \item an initial focus on the software build, integration and test
%   with available robotic vehicles in small scale demonstrations $\sim
%   \$5$--$\$10$ Million for years 1 \& 2.

% \item acquisition of robotic vehicles, buoys, floats and a range of
%   sensors as payloads for these in-situ vehicles, their integration,
%   deployment and demonstration at increasingly larger spatial and
%   temporal scales for $\sim \$20$--$\$30$ Million in years 2 \& 3.

% \item acquisition of funds for a suitable at-scale design, build,
%   test, launch and operation of a \sml constellation with a range of
%   scientific payloads for biological and physical oceanographic
%   measurements for $\sim \$25$ Million in years 4 \& 5.

% \end{itemize}  

% Incremental build and evaluation of this concept can allow us to
% attract a wide range of public and private sponsors in the US and
% Europe.  Equally, we will consistently work with our collaborators in
% the Portuguese government to leverage expensive ship time for testing,
% and other potential in-kind contributions from Portuguese and Spanish
% resources.

% For a long-term operation and viability of this system, multiple
% outcomes can be envisioned. First, with the experience garnered in
% testing and fielding the system, a commercial spin-off of all or parts
% of the technology could be likely. If parts of the technology could be
% monetized and spun off to other companies, \pro can then hold the IP
% while continuing to work on research outcomes after the 4 year
% term. Second, the project can itself look for contracts from
% mega-cities and governments or their agencies to provide a
% software-as-a-service model and be able to subsist as a not-for-profit
% enterprise with unique expertise. Should other private or public
% funding sources be available, those would also be carefully evaluated
% to sustain operations and maintain an R\&D effort.

For a long-term operation and viability of this system, multiple
outcomes can be envisioned. First, with the experience garnered in
testing and fielding the system, a commercial spin-off of all or parts
of the technology could be likely. If parts of the technology could be
monetized and spun off to other companies, \pro can then hold the IP
while continuing to work on research outcomes after the 4-year
term. Second, the non-profit entity that will run the project, can
itself look for contracts from mega-cities and governments or their
agencies to provide a software-as-a-service model and be able to
subsist as a not-for-profit enterprise with unique expertise. Should
other private or public funding sources be available, those would also
be carefully evaluated to sustain operations and maintain an R\&D
effort.


\subsection{Governance and Execution}

The governing board of \pro will consist of prominent strategic
advisors from the US and Europe including stakeholders and funders. In
addition, the project principals will be aided and advised by a
scientific advisory board consisting of technologists, ocean going
scientists, ecologists and policy makers from the US, Europe and
targeted coastal states. 

The project itself will be based in Porto, Portugal for a number of
important reasons including the existence marine robotic development
and testing infrastructure, a skilled tech-savvy work force,
university support and overall cost-effectiveness.

% \newpage
\subsection{The Team}

\proe’s inter-disciplinary team of seasoned researchers (see bio's
below) from the US (SIFT, Columbia and MIT), Portugal and Spain have
worked in all the major oceans, fielded tens of robots at sea
simultaneously, designed/built/flown and operated multiple \smle's and
complex systems in the deep sea and deep space. They have also worked
extensively with one another.

\section{Brief Bios of the Principals}

\parbox{6.5in}{
\begin{wrapfigure}{r}{0.45\textwidth}
  \centering
  \includegraphics[width=.75\linewidth]{fig/FAguado.jpg}
\end{wrapfigure}
\textbf{Fernando Aguado} is an Associate Professor at the University
of Vigo. He was the principal investigator (PI) of the Xatcobeo
cubesat, the first Galician satellite and the first Spanish
cubesat. He was also the PI of HUMSAT-D and sector B of Serpens (both
developed within the Basic Space Technology Initiative of UN with the
support of ESA) and the LUME-1 \smle. He has coordinated the design,
manufacturing and integration of various Cubesats for maritime and
airplane tracking applications as well as for inter-vehicle
communications. 
\\
\textbf{email: }\emph{faguado@tsc.uvigo.es} \\
\textbf{Web: }\url{https://bit.ly/3avJlwX}\\
}

\parbox{6.5in}{
\begin{wrapfigure}{r}{0.45\textwidth}
  \centering
  \includegraphics[width=.75\linewidth]{fig/Pierre.jpg}
\end{wrapfigure}
\textbf{Pierre Lermusiaux} is Professor of Mechanical Engineering and
Ocean Science and Engineering at MIT, and, since July 2018, Associate
Department Head for Research and Operations in Mechanical
Engineering. He has made outstanding contributions in data
assimilation, as well as in ocean modeling and uncertainty
predictions. His research thrusts include understanding and modeling
complex physical and interdisciplinary oceanic dynamics and processes
while utilizing new mathematical models and computational methods for
ocean predictions and dynamical diagnostics for data assimilation
and data-model comparisons.
\\
\textbf{email: }\emph{pierrel@mit.edu}\\
\textbf{Web:}\url{http://web.mit.edu/pierrel/www/} 
}

\parbox{6.5in}{
\begin{wrapfigure}{r}{0.45\textwidth}
  \centering
  \includegraphics[width=.75\linewidth]{fig/Krajan.jpg}
\end{wrapfigure}
\textbf{Kanna Rajan} is a Fellow at SIFT LLC and holds a visiting
faculty appointment at \univ in autonomous systems. He spent 10 years
at \inst where his software was responsible for the command/control of
the 1999 New Millennium Deep Space 1, 65 Million miles from Earth and
the 2003 Mars Exploration Rovers mission on the Red Planet. In 2005 he
moved to \mba and built the only AI group in marine robotics. His
field work and publications have been in highly ranked peer-reviewed
publications including \emph{Science}, Intnl. Journal of Robotics
Research, the AI Journal and Journal of Field Robotics. He has
participated in scientific oceanographic cruises in the Pacific and
Atlantic.
\\
\textbf{email: }\emph{Kanna.Rajan@sift.net}\\
\textbf{Web:}\url{https://www.sift.net/staff/kanna-rajan} }

\vspace*{0.5cm}
\parbox{6.5in}{
\begin{wrapfigure}{r}{0.45\textwidth}
  \centering
  \includegraphics[width=.75\linewidth]{fig/JBS.jpg}
\end{wrapfigure}
\textbf{Jo\~ao Sousa} is a Professor at the Faculty of Engineering,
Univ. of Porto, Portugal and the head of the Underwater Systems and
Technology Laboratory (\lse). The lab has pioneered the design,
construction and deployment of networked underwater, surface and air
vehicles for applications in ocean sciences and defense and is at the
vanguard of operations of coordinated aerial, surface and underwater
vehicles. The lab designed the award-winning Light Autonomous
Underwater Vehicle and the \ls open source software for networked
vehicle systems, and has been key in organizing large scale
experiments, including the annual Rapid Environmental Picture (\rpe)
organized jointly with the Portuguese Navy since 2010. He has
participated and led numerous engineering and scientific oceanographic
cruises.
\\
\textbf{email: }\emph{jtasso@fe.up.pt}\\
\textbf{Web: }\url{https://bit.ly/2J6cKCc}}
% {https://lsts.pt/member/jo%C3%A3o-sousa}

% \vspace*{0.5cm}
\parbox{6.5in}{
\begin{wrapfigure}{r}{0.45\textwidth}
 \centering
  \includegraphics[width=.75\linewidth]{fig/ASub.jpg}
\end{wrapfigure}

\textbf{Ajit Subramaniam} is a Lamont Research Professor at the
Lamont-Doherty Earth Observatory (\ldeoe) of Columbia University, New
York.  He is an oceanographer who uses knowledge of remote sensing,
ocean optics, phytoplankton physiology, biological and physical
oceanography and geographical information systems to better understand
how the marine ecosystem functions and can be managed.  He has worked
for National Oceanic and Atmospheric Administration (NOAA), and held
appointments at the Univ. of Maryland and the Univ. of Southern
California prior to moving \ldeo in 2004. He has served as the Program
Director for the Marine Microbiology Initiative at the Gordon and
Betty Moore Foundation and a program manager in the Biological
Oceanography Program at the U.S. National Science Foundation (NSF). He
has participated in oceanographic cruises in all parts of the world.
\\
\textbf{email: }\emph{ajit@ldeo.columbia.edu} \\
\textbf{Web: }\url{https://www.ldeo.columbia.edu/~ajit/}
}

\vspace*{0.5cm}
\parbox{6.5in}{
\begin{wrapfigure}{r}{0.45\textwidth}
 \centering
  \includegraphics[width=.75\linewidth]{fig/JTintore.jpeg}
\end{wrapfigure}

\textbf{Joaqu\'{i}n Tintor\'{e}} is Professor of Physical Oceanography
and Director of the Spanish Large-Scale Marine Infrastructure SOCIB
(Balearic Islands Coastal Ocean Observing and Forecasting System) that
he proposed and designed in 2006. His present scientific interest is
in understanding ocean state and variability, from episodic events to
climate variability from the coast to the open ocean while
implementing new multi-platform and integrated ocean observing and
forecasting systems.
\\
\textbf{email: }\emph{jtintore@socib.es} \\
\textbf{Web:
}\url{http://www.socib.eu/?seccion=textes&id_textotextes=director} }


\end{document}

