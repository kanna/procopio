\subsection{Sub-project for Keck Foundation support}

\begin{figure}[H]
  \centering
  \includegraphics[scale=0.6]{fig/Audacious-pilot-block-diag-2.pdf}
  \caption{A representation of the technical tasks related to \proe. For
    the \kck the tasks related to model assimilation include portions of
    the boxes at the top of the figure in {\color{blue}{blue}}
    outline. Control aspects relate to using the assimilated ocean model
    to focus where and how autonomous platforms will make observation in
    the coastal ocean.}
    \label{fig:block-diag}
\end{figure}

In the first two years, \pro proposes to focus tasks to those related to
model assimilation and control of autonomous robotic platforms. All
individuals associated with these efforts will be US-based.
Fig. \ref{fig:block-diag} shows the associated tasks highlighted in
{\color{blue}{blue}} and include:

\begin{description}

\item[Automated Data Pipeline, Quality Control \& Assimilation] For
  rapid multi-platform data assimilation and to ensure that the data is
  consistent with expectations of values. This will involve building a
  data pipeline and support infrastructure to ensure that any
  oceanographic sensor or robotic platform can feed data into our ocean
  model ready for assimilation. This task will require a software
  engineer with requisite skills related to data understanding,
  databases, ocean data formats.

\item[Ocean Data Modeler] To make model predictions, \pro will
  leverage off of the MIT Harvard Ocean Prediction
  System\footnote{\url{http://mseas.mit.edu/HOPS/}} model and focus
  the domain of the model to an area where a proposed field experiment
  can demonstrate the concepts in the 'real world'. The task will be
  to initialize the model with the geographical conditions of the
  selected domain including the benthic topology and prime the model
  with remote sensing data adequate to initialize conditions for
  subsequent predictions.

\item[Prediction and Sampling Convergence] Using Bayesian predictive
  capabilities, \pro will pull together a methodology to narrow the
  prediction and sensing gap, between what is sensed in the real-world
  and what was predicted by our model. Strategies to close the expected
  gap will require the use of Machine Learning and Statistical
  techniques, to ensure that robotic platforms can observe in ways to
  bring about convergence. 

\item[Enhancing the Bio-geochemical model] Most models incorporate ocean
  dynamics and the physics related to water flow in the context of
  external forcings such as topology, wind, currents and fresh water
  inflow into the ocean. To make a realistic biological impact
  assessment of these dynamics, the variables associated with nutrients
  transport from the benthic as well as riverine environments, those
  derived from mixing and stirring in the upper water-column as well as
  natural and anthropogenic change need to be incorporated into the
  ocean model to produce plankton functional type responses. This
  requires a specialist who can understand not only the biology, but
  also the physics of nutrient transport and evolution over space and
  time.

\item[Robotic Sampling Algorithms] To ``intelligently'' sample the upper
  water-column, marine robotic vehicles require the implementation of
  algorithms which can sample a wide range of coastal ocean phenomenon
  of interest including Harmful Algal Blooms, plumes, oil slicks and
  hypoxic zones. Using sensor data to craft algorithms embedded on AUVs,
  ASVs and potentially UAVs will require expertise in Statistical
  Sampling and autonomous decision-making and Control. \pro will ensure
  that the latest Sampling methods are embedded for computation on
  robotic vehicles.

\end{description}

\noindent
In addition to supporting the above tasks, \kck funding will help
coordinate work with our international partners ($\sim 10\%$ of budget)
and lay the foundation for demonstrating performance and evaluating the
operational skill of the proposed products and technologies. Some
support for instrumentation for use on robotic platform will also be
requested as part of this request. A near-term goal would be to iterate
between software development and field implementation efforts at the
Univ. of Porto, Portugal.  This collaboration allows for leveraing the
substantial in-situ robotic assets and expertise present there.  Doing
so would allow the team to position \pro to obtain further support to
fulfill the ambition of the entire proposed project in this document.

